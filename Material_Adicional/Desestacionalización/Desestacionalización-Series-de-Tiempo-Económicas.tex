% Options for packages loaded elsewhere
\PassOptionsToPackage{unicode}{hyperref}
\PassOptionsToPackage{hyphens}{url}
%
\documentclass[
]{article}
\usepackage{amsmath,amssymb}
\usepackage{lmodern}
\usepackage{ifxetex,ifluatex}
\ifnum 0\ifxetex 1\fi\ifluatex 1\fi=0 % if pdftex
  \usepackage[T1]{fontenc}
  \usepackage[utf8]{inputenc}
  \usepackage{textcomp} % provide euro and other symbols
\else % if luatex or xetex
  \usepackage{unicode-math}
  \defaultfontfeatures{Scale=MatchLowercase}
  \defaultfontfeatures[\rmfamily]{Ligatures=TeX,Scale=1}
\fi
% Use upquote if available, for straight quotes in verbatim environments
\IfFileExists{upquote.sty}{\usepackage{upquote}}{}
\IfFileExists{microtype.sty}{% use microtype if available
  \usepackage[]{microtype}
  \UseMicrotypeSet[protrusion]{basicmath} % disable protrusion for tt fonts
}{}
\makeatletter
\@ifundefined{KOMAClassName}{% if non-KOMA class
  \IfFileExists{parskip.sty}{%
    \usepackage{parskip}
  }{% else
    \setlength{\parindent}{0pt}
    \setlength{\parskip}{6pt plus 2pt minus 1pt}}
}{% if KOMA class
  \KOMAoptions{parskip=half}}
\makeatother
\usepackage{xcolor}
\IfFileExists{xurl.sty}{\usepackage{xurl}}{} % add URL line breaks if available
\IfFileExists{bookmark.sty}{\usepackage{bookmark}}{\usepackage{hyperref}}
\hypersetup{
  pdftitle={Desestacionalización Series Económicas},
  pdfauthor={Luis Alfonso Luna},
  hidelinks,
  pdfcreator={LaTeX via pandoc}}
\urlstyle{same} % disable monospaced font for URLs
\usepackage[margin=1in]{geometry}
\usepackage{color}
\usepackage{fancyvrb}
\newcommand{\VerbBar}{|}
\newcommand{\VERB}{\Verb[commandchars=\\\{\}]}
\DefineVerbatimEnvironment{Highlighting}{Verbatim}{commandchars=\\\{\}}
% Add ',fontsize=\small' for more characters per line
\usepackage{framed}
\definecolor{shadecolor}{RGB}{248,248,248}
\newenvironment{Shaded}{\begin{snugshade}}{\end{snugshade}}
\newcommand{\AlertTok}[1]{\textcolor[rgb]{0.94,0.16,0.16}{#1}}
\newcommand{\AnnotationTok}[1]{\textcolor[rgb]{0.56,0.35,0.01}{\textbf{\textit{#1}}}}
\newcommand{\AttributeTok}[1]{\textcolor[rgb]{0.77,0.63,0.00}{#1}}
\newcommand{\BaseNTok}[1]{\textcolor[rgb]{0.00,0.00,0.81}{#1}}
\newcommand{\BuiltInTok}[1]{#1}
\newcommand{\CharTok}[1]{\textcolor[rgb]{0.31,0.60,0.02}{#1}}
\newcommand{\CommentTok}[1]{\textcolor[rgb]{0.56,0.35,0.01}{\textit{#1}}}
\newcommand{\CommentVarTok}[1]{\textcolor[rgb]{0.56,0.35,0.01}{\textbf{\textit{#1}}}}
\newcommand{\ConstantTok}[1]{\textcolor[rgb]{0.00,0.00,0.00}{#1}}
\newcommand{\ControlFlowTok}[1]{\textcolor[rgb]{0.13,0.29,0.53}{\textbf{#1}}}
\newcommand{\DataTypeTok}[1]{\textcolor[rgb]{0.13,0.29,0.53}{#1}}
\newcommand{\DecValTok}[1]{\textcolor[rgb]{0.00,0.00,0.81}{#1}}
\newcommand{\DocumentationTok}[1]{\textcolor[rgb]{0.56,0.35,0.01}{\textbf{\textit{#1}}}}
\newcommand{\ErrorTok}[1]{\textcolor[rgb]{0.64,0.00,0.00}{\textbf{#1}}}
\newcommand{\ExtensionTok}[1]{#1}
\newcommand{\FloatTok}[1]{\textcolor[rgb]{0.00,0.00,0.81}{#1}}
\newcommand{\FunctionTok}[1]{\textcolor[rgb]{0.00,0.00,0.00}{#1}}
\newcommand{\ImportTok}[1]{#1}
\newcommand{\InformationTok}[1]{\textcolor[rgb]{0.56,0.35,0.01}{\textbf{\textit{#1}}}}
\newcommand{\KeywordTok}[1]{\textcolor[rgb]{0.13,0.29,0.53}{\textbf{#1}}}
\newcommand{\NormalTok}[1]{#1}
\newcommand{\OperatorTok}[1]{\textcolor[rgb]{0.81,0.36,0.00}{\textbf{#1}}}
\newcommand{\OtherTok}[1]{\textcolor[rgb]{0.56,0.35,0.01}{#1}}
\newcommand{\PreprocessorTok}[1]{\textcolor[rgb]{0.56,0.35,0.01}{\textit{#1}}}
\newcommand{\RegionMarkerTok}[1]{#1}
\newcommand{\SpecialCharTok}[1]{\textcolor[rgb]{0.00,0.00,0.00}{#1}}
\newcommand{\SpecialStringTok}[1]{\textcolor[rgb]{0.31,0.60,0.02}{#1}}
\newcommand{\StringTok}[1]{\textcolor[rgb]{0.31,0.60,0.02}{#1}}
\newcommand{\VariableTok}[1]{\textcolor[rgb]{0.00,0.00,0.00}{#1}}
\newcommand{\VerbatimStringTok}[1]{\textcolor[rgb]{0.31,0.60,0.02}{#1}}
\newcommand{\WarningTok}[1]{\textcolor[rgb]{0.56,0.35,0.01}{\textbf{\textit{#1}}}}
\usepackage{graphicx}
\makeatletter
\def\maxwidth{\ifdim\Gin@nat@width>\linewidth\linewidth\else\Gin@nat@width\fi}
\def\maxheight{\ifdim\Gin@nat@height>\textheight\textheight\else\Gin@nat@height\fi}
\makeatother
% Scale images if necessary, so that they will not overflow the page
% margins by default, and it is still possible to overwrite the defaults
% using explicit options in \includegraphics[width, height, ...]{}
\setkeys{Gin}{width=\maxwidth,height=\maxheight,keepaspectratio}
% Set default figure placement to htbp
\makeatletter
\def\fps@figure{htbp}
\makeatother
\setlength{\emergencystretch}{3em} % prevent overfull lines
\providecommand{\tightlist}{%
  \setlength{\itemsep}{0pt}\setlength{\parskip}{0pt}}
\setcounter{secnumdepth}{-\maxdimen} % remove section numbering
\ifluatex
  \usepackage{selnolig}  % disable illegal ligatures
\fi

\title{Desestacionalización Series Económicas}
\author{Luis Alfonso Luna}
\date{2021 - 2}

\begin{document}
\maketitle

\begin{Shaded}
\begin{Highlighting}[]
\FunctionTok{library}\NormalTok{(forecast) }\CommentTok{\#paquete más reciente disponible (fable)}
\FunctionTok{library}\NormalTok{(seasonal) }\CommentTok{\#paquete más reciente disponible (fabletools)}
\FunctionTok{library}\NormalTok{(quantmod) }
\FunctionTok{library}\NormalTok{(tidyverse)}
\end{Highlighting}
\end{Shaded}

\hypertarget{cargar-series-de-tiempo-macro-y-financieras-directamente-en-r}{%
\section{Cargar Series de Tiempo Macro y financieras directamente en
R}\label{cargar-series-de-tiempo-macro-y-financieras-directamente-en-r}}

Usando el paquete QuantMod (Ryan \& Ulrich, 2020), podemos cargar
directamente series de tiempo cuya fuente sea Yahoo Finance, FRED,
MySQL, Csv, RData, Oanda, o Av.

\begin{Shaded}
\begin{Highlighting}[]
\FunctionTok{getSymbols}\NormalTok{(}\StringTok{"UNRATE"}\NormalTok{,}\AttributeTok{src =} \StringTok{"FRED"}\NormalTok{)}
\end{Highlighting}
\end{Shaded}

\begin{verbatim}
## [1] "UNRATE"
\end{verbatim}

\begin{Shaded}
\begin{Highlighting}[]
\FunctionTok{plot}\NormalTok{(UNRATE)}
\end{Highlighting}
\end{Shaded}

\includegraphics{Desestacionalización-Series-de-Tiempo-Económicas_files/figure-latex/unnamed-chunk-2-1.pdf}

\begin{Shaded}
\begin{Highlighting}[]
\FunctionTok{plot}\NormalTok{(UNRATE, }\AttributeTok{subset =} \StringTok{"2010{-}01/2021{-}03"}\NormalTok{)}
\end{Highlighting}
\end{Shaded}

\includegraphics{Desestacionalización-Series-de-Tiempo-Económicas_files/figure-latex/unnamed-chunk-3-1.pdf}

\begin{Shaded}
\begin{Highlighting}[]
\FunctionTok{getSymbols}\NormalTok{(}\StringTok{"AAPL"}\NormalTok{, }\AttributeTok{src =} \StringTok{"yahoo"}\NormalTok{)}
\end{Highlighting}
\end{Shaded}

\begin{verbatim}
## [1] "AAPL"
\end{verbatim}

\begin{Shaded}
\begin{Highlighting}[]
\FunctionTok{plot}\NormalTok{(AAPL)}
\end{Highlighting}
\end{Shaded}

\includegraphics{Desestacionalización-Series-de-Tiempo-Económicas_files/figure-latex/unnamed-chunk-4-1.pdf}

\begin{Shaded}
\begin{Highlighting}[]
\FunctionTok{chartSeries}\NormalTok{(AAPL)}
\end{Highlighting}
\end{Shaded}

\includegraphics{Desestacionalización-Series-de-Tiempo-Económicas_files/figure-latex/unnamed-chunk-5-1.pdf}

\hypertarget{descomposiciuxf3n-series-de-tiempo}{%
\section{Descomposición series de
tiempo}\label{descomposiciuxf3n-series-de-tiempo}}

Toda serie de tiempo puede dividirse en al menos tres componentes:
tendencia, estacionalidad (\textbf{no confundir con estacionareidad}) y
ciclos. Usualmente se mezcla tendencia y ciclos, por lo que podemos
descompner nuestras series de tiempo en:

\begin{itemize}
\tightlist
\item
  Tendencia-ciclo (a veces llamado solamente tendencia).
\item
  Estacional.
\item
  Todo lo demás (a veces llamado término error, \textbf{ojo}: no es el
  mismo error de la regresión, no confundir; evitaremos ese nombre).
\end{itemize}

Si asumimos una composición aditiva, podemos representar cualquier serie
de tiempo como:

\[y_{t} = S_{t} + T_{t} + R_t,\]

Donde \(y_t\) es la serie, \(S_t\) es la estacionalidad, \(T_t\) es la
tendencia-ciclo y \(R_t\) es todo lo demás (remainder). Podemos asumir
también una composición multiplicativa.

\[y_{t} = S_{t} \times T_{t} \times R_t,\]

Es correcto asumir una composición aditiva cuando la magnitud de las
fluctuaciones estacionales no varía con el nivel de la serie de tiempo.
Cuando la variación en \(S_{t}\) y/o \(T_{t}\) parece ser proporcional
al nivel de la serie de tiempo, usamos multiplicativa. Usualmente se
asume composición multiplicativa en series de tiempo económicas.

Una alternativa a hacer descomposición multiplicativa es estabilizar los
datos y luego asumir composición aditiva. Cuando la transformación es un
logaritmo asumimos que es equivalente a hacer descomposición
multiplicativa:

\[y_{t} = S_{t} \times T_{t} \times R_t \quad\text{es equivalente a}\quad
  \log y_{t} = \log S_{t} + \log T_{t} + \log R_t.\]

Existen múltiples métodos para descomponer una serie, entre estos:

\begin{itemize}
\tightlist
\item
  Clásico.
\item
  X11.
\item
  Seats.
\item
  STL.
\end{itemize}

La descomposición es útil porque nos permite hacer mejores pronósticos,
entender que hay debajo de la serie y encontrar algunas variables
económicas (ej. PIB potencial).

\hypertarget{muxe9todo-cluxe1sico-multiplicativo}{%
\subsubsection{Método clásico
(multiplicativo)}\label{muxe9todo-cluxe1sico-multiplicativo}}

\begin{enumerate}
\def\labelenumi{\arabic{enumi}.}
\item
  Si \(m\) es un número par, encuentre la tendencia ciclo
  \(\hat{T}_{t}\) usando \(2 \times m\) -MA. Si \(m\) es impar,
  encuentre la tendencia ciclo \(\hat{T}_{t}\) usando \(m-\mathrm{MA}\).
\item
  Calcule la serie sin tendencia: \(y_t - \hat{T}_t\).
\item
  Para estimar el componente estacional de cada temporada, encuentre el
  promedio para esa temporada. Por ejemplo, si tiene valores mensuales,
  el componente estacional de marzo es el promedio de todos los valores
  de marzo en los datos. Esto da \(\hat{S}_t\).
\item
  El resto lo obtiene substrayendo:
  \(\hat{R}_t = y_t - \hat{T}_t - \hat{S}_t\).
\end{enumerate}

\begin{Shaded}
\begin{Highlighting}[]
\FunctionTok{getSymbols}\NormalTok{(}\StringTok{"COLLRUNTTTTSTM"}\NormalTok{,}\AttributeTok{src=}\StringTok{"FRED"}\NormalTok{)}
\end{Highlighting}
\end{Shaded}

\begin{verbatim}
## [1] "COLLRUNTTTTSTM"
\end{verbatim}

\begin{Shaded}
\begin{Highlighting}[]
\NormalTok{td }\OtherTok{\textless{}{-}}\NormalTok{ COLLRUNTTTTSTM}
\NormalTok{td }\OtherTok{\textless{}{-}}\NormalTok{ td[}\StringTok{"/2020{-}05"}\NormalTok{]}
\NormalTok{td }\OtherTok{\textless{}{-}} \FunctionTok{ts}\NormalTok{(td,}\AttributeTok{start =} \FunctionTok{c}\NormalTok{(}\DecValTok{2007}\NormalTok{,}\DecValTok{01}\NormalTok{),}\AttributeTok{frequency =} \DecValTok{12}\NormalTok{)}
\FunctionTok{autoplot}\NormalTok{(td)}
\end{Highlighting}
\end{Shaded}

\includegraphics{Desestacionalización-Series-de-Tiempo-Económicas_files/figure-latex/unnamed-chunk-6-1.pdf}

\begin{Shaded}
\begin{Highlighting}[]
\NormalTok{td }\SpecialCharTok{\%\textgreater{}\%} \FunctionTok{decompose}\NormalTok{(}\AttributeTok{type=}\StringTok{"multiplicative"}\NormalTok{) }\SpecialCharTok{\%\textgreater{}\%}
  \FunctionTok{autoplot}\NormalTok{() }\SpecialCharTok{+} \FunctionTok{xlab}\NormalTok{(}\StringTok{"Año"}\NormalTok{) }\SpecialCharTok{+}
  \FunctionTok{ggtitle}\NormalTok{(}\StringTok{"Descomposión multiplicativa clásica de la Tasa de Desempleo"}\NormalTok{)}
\end{Highlighting}
\end{Shaded}

\includegraphics{Desestacionalización-Series-de-Tiempo-Económicas_files/figure-latex/unnamed-chunk-7-1.pdf}

Tener en cuenta que:

\begin{itemize}
\tightlist
\item
  El cálculo de la tendencia no está disponible para los últimos y los
  primeros periodos.
\item
  La tendencia suele suavizar mucho las caídas y subidas rápidas en los
  datos.
\item
  Se asume que el componente cíclico se repite de año a año.
\item
  Mal método para recoger cambios en periodos cortos de tiempo.
\end{itemize}

\hypertarget{muxe9todo-x11}{%
\subsection{Método X11}\label{muxe9todo-x11}}

Es el método más usado, desarrollado en US Census Bureau. Está basado en
el método clásico, pero corrige los inconvenientes expuestos
anteriormente. Para más información mirar Dagum \& Bianconcini (2016).

\begin{Shaded}
\begin{Highlighting}[]
\NormalTok{td }\SpecialCharTok{\%\textgreater{}\%} \FunctionTok{seas}\NormalTok{(}\AttributeTok{x11 =} \StringTok{""}\NormalTok{) }\OtherTok{{-}\textgreater{}}\NormalTok{ fit}
\FunctionTok{autoplot}\NormalTok{(fit) }\SpecialCharTok{+}
  \FunctionTok{ggtitle}\NormalTok{(}\StringTok{"Descomposión X11 de la Tasa de Desempleo"}\NormalTok{)}
\end{Highlighting}
\end{Shaded}

\includegraphics{Desestacionalización-Series-de-Tiempo-Económicas_files/figure-latex/unnamed-chunk-8-1.pdf}

Se puede ver en el reminder, que el método X11 captura mejor los datos
atípicos. Note también que el componente cíclico del ajuste estacional
no es igual año a año.

\begin{Shaded}
\begin{Highlighting}[]
\FunctionTok{autoplot}\NormalTok{(td, }\AttributeTok{series=}\StringTok{"Tasa de desempleo"}\NormalTok{) }\SpecialCharTok{+}
  \FunctionTok{autolayer}\NormalTok{(}\FunctionTok{trendcycle}\NormalTok{(fit), }\AttributeTok{series=}\StringTok{"Tendencia"}\NormalTok{) }\SpecialCharTok{+}
  \FunctionTok{autolayer}\NormalTok{(}\FunctionTok{seasadj}\NormalTok{(fit), }\AttributeTok{series=}\StringTok{"Ajuste Estacional"}\NormalTok{) }\SpecialCharTok{+}
  \FunctionTok{xlab}\NormalTok{(}\StringTok{"Año"}\NormalTok{) }\SpecialCharTok{+} \FunctionTok{ylab}\NormalTok{(}\StringTok{""}\NormalTok{) }\SpecialCharTok{+}
  \FunctionTok{ggtitle}\NormalTok{(}\StringTok{"Tasa Desempleo Colombia"}\NormalTok{) }\SpecialCharTok{+}
  \FunctionTok{scale\_colour\_manual}\NormalTok{(}\AttributeTok{values=}\FunctionTok{c}\NormalTok{(}\StringTok{"gray"}\NormalTok{,}\StringTok{"blue"}\NormalTok{,}\StringTok{"red"}\NormalTok{),}
             \AttributeTok{breaks=}\FunctionTok{c}\NormalTok{(}\StringTok{"Tasa de desempleo"}\NormalTok{,}\StringTok{"Ajuste Estacional"}\NormalTok{,}\StringTok{"Tendencia"}\NormalTok{))}
\end{Highlighting}
\end{Shaded}

\includegraphics{Desestacionalización-Series-de-Tiempo-Económicas_files/figure-latex/unnamed-chunk-9-1.pdf}

Puede ser hacer subgráficos mensuales del componente estacional. Así
puede observarse los cambios en el componente estacional a través del
tiempo.

\begin{Shaded}
\begin{Highlighting}[]
\NormalTok{fit }\SpecialCharTok{\%\textgreater{}\%} \FunctionTok{seasonal}\NormalTok{() }\SpecialCharTok{\%\textgreater{}\%} \FunctionTok{ggsubseriesplot}\NormalTok{() }\SpecialCharTok{+} \FunctionTok{ylab}\NormalTok{(}\StringTok{"Componente Estacional"}\NormalTok{)}
\end{Highlighting}
\end{Shaded}

\includegraphics{Desestacionalización-Series-de-Tiempo-Económicas_files/figure-latex/unnamed-chunk-10-1.pdf}

\hypertarget{descomposiciuxf3n-seats}{%
\subsection{Descomposición SEATS}\label{descomposiciuxf3n-seats}}

SEATS significa ``Seasonal Extraction in ARIMA Time Series''.
Desarrollado por el Banco de España, hoy es uno de los métodos más
usados. Desgraciadamente solamente funciona con datos trimestrales y
mensuales. En la práctica da resultados muy similares a X11.

\begin{Shaded}
\begin{Highlighting}[]
\NormalTok{td }\SpecialCharTok{\%\textgreater{}\%} \FunctionTok{seas}\NormalTok{() }\SpecialCharTok{\%\textgreater{}\%}
\FunctionTok{autoplot}\NormalTok{() }\SpecialCharTok{+}
  \FunctionTok{ggtitle}\NormalTok{(}\StringTok{"Descomposión SEATS de la Tasa de Desempleo"}\NormalTok{)}
\end{Highlighting}
\end{Shaded}

\includegraphics{Desestacionalización-Series-de-Tiempo-Económicas_files/figure-latex/unnamed-chunk-11-1.pdf}

\hypertarget{descomposiciuxf3n-stl}{%
\subsection{Descomposición STL}\label{descomposiciuxf3n-stl}}

Acrónimo de ``Seasonal and Trend decomposition using Loess'',
desarrollado por Cleveland, Cleveland, McRae, \& Terpenning (1990).
Tiene varias ventajas respecto a X11 y SEATS.

\begin{itemize}
\tightlist
\item
  Puede trabajar con cualquier tipo de estacionalidad.
\item
  El investigador puede controlar manualmente el componente de cambio
  estacional.
\item
  El suavizado de la tendencia también puede ser controlado manualmente.
\item
  Robusto a datos atípicos.
\end{itemize}

Sin embargo hay que tener cuidado, ya que la utilidad del método reside
completamente en los parámetros manuales.

Los dos parámetros a elegir son la ``ventana'' tendencia-ciclo y la
``ventana'' estacional. Valores más pequeños permiten cambios más
rápidos. Ambos parámetros deben ser números impares, donde t.window es
el número de observaciones consecutivas que se usan al estimar la
tendencia-ciclo; s.window es el número de años consecutivos del
componente estacional.

\begin{Shaded}
\begin{Highlighting}[]
\NormalTok{td[,}\DecValTok{1}\NormalTok{] }\SpecialCharTok{\%\textgreater{}\%}
  \FunctionTok{stl}\NormalTok{(}\AttributeTok{t.window=}\DecValTok{13}\NormalTok{, }\AttributeTok{s.window=}\StringTok{"periodic"}\NormalTok{, }\AttributeTok{robust=}\ConstantTok{TRUE}\NormalTok{) }\SpecialCharTok{\%\textgreater{}\%}
  \FunctionTok{autoplot}\NormalTok{()}
\end{Highlighting}
\end{Shaded}

\includegraphics{Desestacionalización-Series-de-Tiempo-Económicas_files/figure-latex/unnamed-chunk-12-1.pdf}

Puedo usar mstl() para que haga la composición de manera automática. En
este caso no hay mucha diferencia con X11.

\begin{Shaded}
\begin{Highlighting}[]
\NormalTok{td }\SpecialCharTok{\%\textgreater{}\%}
  \FunctionTok{mstl}\NormalTok{() }\SpecialCharTok{\%\textgreater{}\%}
  \FunctionTok{autoplot}\NormalTok{()}
\end{Highlighting}
\end{Shaded}

\includegraphics{Desestacionalización-Series-de-Tiempo-Económicas_files/figure-latex/unnamed-chunk-13-1.pdf}

Para obtener la serie desetacionalizada basta con usar la función
\emph{final()}.

\begin{Shaded}
\begin{Highlighting}[]
\NormalTok{td\_desest }\OtherTok{\textless{}{-}}\NormalTok{ td }\SpecialCharTok{\%\textgreater{}\%} \FunctionTok{seas}\NormalTok{(}\AttributeTok{x11=}\StringTok{""}\NormalTok{) }\SpecialCharTok{\%\textgreater{}\%} \FunctionTok{final}\NormalTok{()}
\FunctionTok{plot}\NormalTok{(td\_desest)}
\end{Highlighting}
\end{Shaded}

\includegraphics{Desestacionalización-Series-de-Tiempo-Económicas_files/figure-latex/unnamed-chunk-14-1.pdf}

\begin{Shaded}
\begin{Highlighting}[]
\FunctionTok{autoplot}\NormalTok{(td\_desest) }\SpecialCharTok{+} \FunctionTok{autolayer}\NormalTok{(td)}
\end{Highlighting}
\end{Shaded}

\includegraphics{Desestacionalización-Series-de-Tiempo-Económicas_files/figure-latex/unnamed-chunk-15-1.pdf}

\begin{Shaded}
\begin{Highlighting}[]
\NormalTok{pronost }\OtherTok{\textless{}{-}} \FunctionTok{stlf}\NormalTok{(td, }\AttributeTok{method =} \StringTok{"naive"}\NormalTok{)}
\FunctionTok{autoplot}\NormalTok{(pronost)}
\end{Highlighting}
\end{Shaded}

\includegraphics{Desestacionalización-Series-de-Tiempo-Económicas_files/figure-latex/unnamed-chunk-16-1.pdf}

\end{document}
